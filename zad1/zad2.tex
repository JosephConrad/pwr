
%----------------------------------------------------------------------------------------
%	PACKAGES AND OTHER DOCUMENT CONFIGURATIONS
%----------------------------------------------------------------------------------------

\documentclass[oneside]{book}
\usepackage{polski}
\usepackage{graphicx}
\usepackage{tabularx}
\usepackage{amsmath,amssymb} 
\usepackage{fancyhdr}
\usepackage{algorithm}
\usepackage{algpseudocode}
\usepackage{hyperref}
\usepackage[margin=1in]{geometry}
\usepackage[svgnames, table]{xcolor} % Required to specify font color

\usepackage{fancyhdr}
\usepackage[utf8]{inputenc}

% do tabel
\usepackage[T1]{fontenc}
\usepackage[utf8]{inputenc}
\usepackage{tabularx,ragged2e,booktabs,caption}
\newcolumntype{C}[1]{>{\Centering}m{#1}}
\renewcommand\tabularxcolumn[1]{C{#1}}
\renewcommand{\listalgorithmname}{List of \ALG@name s}



\makeatletter 
\makeatother

\newcommand*{\plogo}{\fbox{$\mathcal{PL}$}} % Generic publisher logo

%----------------------------------------------------------------------------------------
%	TITLE PAGE
%----------------------------------------------------------------------------------------

\newcommand*{\titleTH}{\begingroup % Create the command for including the title page in the document
\raggedleft % Right-align all text
\vspace*{\baselineskip} % Whitespace at the top of the page

{\Large Konrad Lisiecki 291649}\\[0.167\textheight] % Author name 

{\LARGE\bfseries Zadanie 1 - Algorytm Szymańskiego}\\[\baselineskip] % First part of the title, if it is unimportant consider making the font size smaller to accentuate the main title

{\textcolor{Red}{\Huge Programowanie współbieżne i Rozproszone}}\\[\baselineskip] % Main title which draws the focus of the reader

{\Large \textit{Semestr letni 2013/14  }}\par % Tagline or further description

\vfill % Whitespace between the title block and the publisher

{\large  Warszawa, 2014-03-17}\par % Publisher and logo
\renewcommand{\chaptername}{Podpunkt}

\vspace*{3\baselineskip} % Whitespace at the bottom of the page
\endgroup}


\newcolumntype{R}{>{\raggedleft\arraybackslash}X}
%----------------------------------------------------------------------------------------
%	BLANK DOCUMENT
%----------------------------------------------------------------------------------------

\begin{document} 

\pagestyle{empty} % Removes page numbers
	

\titleTH % This command includes the title page


\tableofcontents
%\listoffigures
%\listoftables

%\chapter*{Wprowadzenie}
%\addcontentsline{toc}{chapter}{Wprowadzenie}

\chapter{Sumaryczne wyniki przedstawione w postaci tabeli}
\paragraph{Własności algorytmów} Czy odpowiedni wariant algorytmu Szymańskiego ma daną własność:

\begin{table}[H]                                                                                                                                                             
  \begin{tabularx}{\textwidth}{| l | R | R | R | R | R | R | R |}                                                      
   \hline                                                                            \input{dane} \hline
   \end{tabularx}                                                                
  \caption{Własności wariantów algorytmu Szymańskiego}                 
 \end{table}        



\chapter{Sprawdzenie własności protokołu}


Celem zadania jest zaimplementowanie Algorytmu Szymańskiego, który pozwala na zachowanie wzajemnego wykluczania dla więcej niż $2$ procesów. Po zaimplementowaniu algorytmu można, za pomocą logiki $LTL$, sprawdzać wiele ciekawych własności dotyczących przebiegu wykonania procesów. Wśród nich możemy wyróżnić:
\begin{enumerate}
\item Wzajemne wykluczanie
\item Nieunikniona poczekalnia
\item Wyjście z poczekalni
\item Żywotność
\item Liniowy czas oczekiwania
\end{enumerate}
W kolejnych podpunktach rozdziału przyjrzymy się bliżej tym własnościom.

 

\section{Wzajemne wykluczanie} - program \textbf{MA} własność wzajemnego wykluczania. (w całym tekście, gdy program ma jakąś własność sprawdzamy to przez przejście i sprawdzenie wszystkich możliwych stanów). Formuła logiki $LTL$ wygląda w następujący sposób:



% Wzajemne wykluczanie

 \begin{algorithm}[H]                                                            
   \caption{Wzajemne Wykluczanie}\label{euclid}
   \begin{algorithmic}[1]                                                         
     \State $ltl \; sk \; { \square (counter < 2)}$\Comment{Formuła LTL dla wzajemnego wykluczania} 
  \end{algorithmic}                                                             
\end{algorithm}   



% Nieunikniona poczekalnia

\section{Nieunikniona poczekalnia} (każde wejście do sekcji krytycznej procesu i poprzedzone jest przez stan o własności we[i] \&\& !chce[i]) - program \textbf{NIE MA} własności nieuniknionej poczekalni. Formuła logki LTL wygląda następująco:
 \begin{algorithm}[H]                                                            
   \caption{Nieunikniona poczekalnia}\label{euclid}
   \begin{algorithmic}[1]                                                         
     \State $ltl \; np \; { \square (P[1]@sk \rightarrow nieunikniona\_poczekalnia[1])} $\Comment{Formuła LTL dla nieuniknionej poczekalni} 
  \end{algorithmic}                                                             
\end{algorithm} 

\paragraph{Komentarz do braku nieuniknionej poczekalni:}
Kiedy mamy tylko jeden proces, który chce wejść do sekcji krytycznej, to wchodzi tam nie przechodząc przez poczekalnię.






% Wyjście z poczekalni

\section{Wyjście z poczekalni} (jeśli któryś z procesów i ma własność we[i] \&\& !chce[i], to w końcu któryś z pozostałych procesów j będzie miał własność wy[j]) - program \textbf{MA} własność wyjścia z poczekalni. Formuła logiki LTL wygląda jak poniżej:



 \begin{algorithm}[H]                                                            
   \caption{Wyjście z poczekalni}\label{euclid}
   \begin{algorithmic}[1]                                                         
     \State $ltl \;wyj\; { \square (nieunikniona\_poczekalnia[0] \rightarrow \lozenge P[1]@tutaj\; || \;\lozenge P[2]@tutaj \;|| \; \lozenge P[3]@tutaj)    } $\Comment{Formuła LTL dla wyjścia z poczekalni} 
  \end{algorithmic}                                                             
\end{algorithm} 




% Żywotnosc


\section{Żywotność}
Program \textbf{NIE MA} własności żywotności. Formuła logiki LTL wygląda jak poniżej:
\begin{algorithm}[H]                                                            
   \caption{Żywotność}\label{euclid}
   \begin{algorithmic}[1]                                                         
     \State $ltl \; zyw \; { \square (P[i]@przed\_pocz \rightarrow \lozenge P[i]@sk) }  $\Comment{Formuła LTL dla żywotności} 
  \end{algorithmic}                                                             
\end{algorithm} 


\paragraph{Komentarz do braku żywotności:}
Rozpatrzmy następujący scenariusz: Załóżmy, że procesy $3$ i $4$ nie wyrażając chęci znalezienia się w sekcji krytycznej. W pewnym momencie proces $1$ wykonuje cały przebieg algorytmu i zatrzymuje się dokładnie przed przypisaniem $chce[i]=false$ (instrukcja ta ma numer 16). Następnie procesor otrzymuje proces $2$ i musi znaleźć się w poczekalni. Kolejno następuje przełączenie kontekstu, proces $1$ kończy protokół końcowy i nie wyraża już zainteresowania znalezieniem się w sekcji krytycznej (pętli się we fragmencie własne sprawy). W ten sposób proces $2$ nigdy nie wejdzie do sekcji krytycznej.



% Liniowy czas oczekiwania

\section{Liniowy czas oczekiwania} (podczas gdy jakiś proces czeka, żaden inny proces nie może wejść do sekcji krytycznej więcej niż stałą liczbę razy)- program \textbf{MA} własność liniowości czasu oczekiwania. Formuła LTL została wyrażona się jak poniżej:
 \begin{algorithm}[H]                                                            
   \caption{Liniowy czas oczekiwania}\label{euclid}
   \begin{algorithmic}[1]                                                         
     \State $ltl \; lco \; {\square(licznik[0] \leq N) }  $\Comment{Formuła LTL dla liniowego czasu oczekiwania} 
  \end{algorithmic}                                                             
\end{algorithm} 
 
 \paragraph{Komentarz do tablicy licznik:} Tablica licznik trzyma liczbę procesów, które weszły do sekcji krytycznej podczas oczekiwania przez dany proces.




\chapter{Odporność własności algorytmu na zmianę kolejności przypisań}

W celu sprawdzenia jak zachowują się poszczególne wersje algorytmu najpierw sprawdzałem jakie będą wyniki w przypadku gdy omawiane trzy istrukcje programu zostaną wykonane atomowo:
 \begin{algorithm}[H]                                                            
   \caption{Atmomowe wykonanie $3$ ostatnich instrukcji programu}\label{euclid}
   \begin{algorithmic}[1]                                                          
            \State atomic                                                              
           \State $\{$                                                                   
                \State  \hspace{\algorithmicindent} wy[i] = false;                                                  
                \State \hspace{\algorithmicindent} we[i] = false;                                                  
                 \State \hspace{\algorithmicindent} chce[i] = false;  
				\State$\}$
				
				  \end{algorithmic}                                                             
\end{algorithm}   

Po wprowadzeniu atomowości okazało się, że algorytm zyskał jedną własność, której poprzednio, nie miał, a mianowicie żywotność:

\begin{table}[H]                                                                                                                                                             
\centering
  \begin{tabular}{| l |r |}                                                      
   \hline                                                                            \input{liveliness} \hline
   \end{tabular}                                                                
  \caption{Własności wariantów algorytmu Szymańskiego dla zatomizowanych ostatnich instrukcji}                 
 \end{table} 
 
Z tego wynika, że musi istnieć inna kolejność wykonań, taka, że zajdzie własność żywotności. Okazało się, że permutacja $(14,16,15)$ zapewnia tę własność. Co więcej, wynikiem sprawdzenia pozostałych własności są 2 obserwacje:
\begin{enumerate}
\item Permutacja $(14,16,15)$ spełnia wszystkie własności oprócz nieuniknionej
poczekalni. Oznacza to, że jest lepszym wariantem algotymu od tego, który był
pierwotnie w zadaniu. Permutacja te spełnia własność, ponieważ przypisanie $chce[i] = false$ odbywa się przed przypisaniem $we[i] = false$. Wtedy to inny proces sprawdzający warunek na ominięcie poczekalni (druga pętla) otrzyma "zgode" na przejście od razu za poczekalnie.
\item Pozostałe permutacje (tzn. takie, dla których $14$ instrukcja nie jest początkową) nie spełniają żadnych instrukcj, włączając w to własność wzajemnego wykluczania.
\end{enumerate}

Wyniki analizay zostały przedstawione w poniższej tabeli:

\begin{table}[H]                                                                                                                                                             
\centering
  \begin{tabularx}{\textwidth}{| l | R | R | R | R | R | R |}                                                      
   \hline                                                                            \input{perm} \hline
   \end{tabularx}                                                                
  \caption{Własności wariantów algorytmu Szymańskiego dla poszczególnych permutacji ostatnich $3$ instrukcji.}                 
 \end{table} 





\chapter{Możliwość "awarii" procesów, a własności algorytmu - pkt 5}
W przypadku gdy istnieje możliwość awarii procesu, proces kończy swój normalny bieg i wraca do początku swojego wykonania. W przypadku zmodyfikowania modelu, jak w punkcie $4$ zadania, algortym nie spełnia żadnych z własności, włącznie ze wzajemnym wykluczaniem. Po zaimplementowaniu tej części, postanowiłem odpowidnio zmodyfikować algorytm (zgodnie z załączonym artykułem o tym algorytmie). Po zmianie udało się uzyskać własność wzajmengo wykluczania. Implementacja znajduje się w pliku $prot2.pml$.\\

 
\begin{table}[H]                                                                                                                                                             
\centering
  \begin{tabular}{| l |r |}                                                      
   \hline                                                                            \input{breakdown} \hline
   \end{tabular}                                                                
  \caption{Własności algorytmu z możliwością "awarii".}                 
 \end{table} 

\paragraph{Własności algorymu z awariami} Jak zostało wspomniane wzajemne wykluczanie jest spełnione, natomiast pozostałe nie są spełnione. Jeśli chodzi o nieuniknioną poczekalnię i żywotność, to argumenty dlaczego własności te nie są spełnione są takie same jak w rozdziale drugim (bo może się zdarzyć, że procesy się nigdy nie będą psuły - czyli możemy powiedzieć, że oryginalny algorytm Szymańskiego jest szczególnym przypadkiem jego wersji z awariami). Jeżeli chodzi o własność trzecią (wyjście z poczekalni) to sprawa wydaje się oczywista, gdy uwzględnimy to, że procesy są podatne na awarie. Z tego wynika, że może zaistnieć sytuacja, kiedy procesy będą się "psuć" przed przypisaniem $wy[i] = true$ (instrukcja 13). Również dla liniowego czasu oczekiwania weryfikacja wykazała, że ta własność nie jest spełniona.

\end{document}
