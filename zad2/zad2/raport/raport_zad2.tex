
%----------------------------------------------------------------------------------------
%	PACKAGES AND OTHER DOCUMENT CONFIGURATIONS
%----------------------------------------------------------------------------------------

\documentclass[oneside]{book} 
\usepackage{polski}
\usepackage{graphicx}
\usepackage{tabularx}
\usepackage{amsmath,amssymb} 
\usepackage{fancyhdr}
\usepackage{algorithm}
\usepackage{algpseudocode}
\usepackage{hyperref}
\usepackage[margin=1in]{geometry}
\usepackage[svgnames, table]{xcolor} % Required to specify font color

\usepackage{fancyhdr}
\usepackage[utf8]{inputenc}

% do tabel
\usepackage[T1]{fontenc}
\usepackage[utf8]{inputenc}
\usepackage{tabularx,ragged2e,booktabs,caption}
\newcolumntype{C}[1]{>{\Centering}m{#1}}
\renewcommand\tabularxcolumn[1]{C{#1}}
\renewcommand{\listalgorithmname}{List of \ALG@name s}



\makeatletter
\makeatother

\newcommand*{\plogo}{\fbox{$\mathcal{PL}$}} % Generic publisher logo

%----------------------------------------------------------------------------------------
%	TITLE PAGE
%----------------------------------------------------------------------------------------

\newcommand*{\titleTH}{\begingroup % Create the command for including the title page in the document
\raggedleft % Right-align all text
\vspace*{\baselineskip} % Whitespace at the top of the page

{\Large Konrad Lisiecki 291649}\\[0.167\textheight] % Author name 

{\LARGE\bfseries Zadanie 2 - Algorytm Brandesa}\\[\baselineskip] % First part of the title, if it is unimportant consider making the font size smaller to accentuate the main title

{\textcolor{Red}{\Huge Programowanie współbieżne i Rozproszone}}\\[\baselineskip] % Main title which draws the focus of the reader

{\Large \textit{Semestr letni 2013/14  }}\par % Tagline or further description

\vfill % Whitespace between the title block and the publisher

{\large  Warszawa, 2014-04-27}\par % Publisher and logo
\renewcommand{\chaptername}{Podpunkt}

\vspace*{3\baselineskip} % Whitespace at the bottom of the page
\endgroup}


\newcolumntype{R}{>{\raggedleft\arraybackslash}X}
%----------------------------------------------------------------------------------------
%	BLANK DOCUMENT
%----------------------------------------------------------------------------------------

\begin{document} 

\pagestyle{empty} % Removes page numbers
	

\titleTH % This command includes the title page


\tableofcontents
%\listoffigures
%\listoftables

%\chapter*{Wprowadzenie}
%\addcontentsline{toc}{chapter}{Wprowadzenie}

\chapter{Wprowadzenie}
\paragraph{Algorytm Brandesa} 

Celem zadania jest zaimplementowanie Algorytmu Brandesa. Algorytm ten jest używany w zagadnieniach związanych z analizą sieci, a dokładniej rzecz ujmując, z identyfikacją ważnych wierzchołków w sieciach. Może on być stosowany do wyszukiwania istotnych elementów w sieciach nieważonych, tzn. takich dla których nie przypisujemy wag dla połączenia dwóch danych składników sieci. Jest to istotne założenie, ponieważ dzięki temu algorytm brandesa osiąga znacznie lepszą złożoność obliczeniową ($O|V||E|$) w porównaniu do najszybszych
algorytmów działających na sieciach ważonych ($O |V|^3$ dla alg. Floyda-Warshalla). Jednak mimo tego ograniczenia algorytm Brandesa świetnie najdaje się do analizy wielu powszechnie występujących sieci, jak np. sieci społecznościowych.


\paragraph{Użyte jednostki obliczeniowe} Przedstawiony powyżej algorym został zaimplementowany na dwóch jednostkach obliczeniowych, mianowicie na procesorze graficznym GPU i procesorze jednostki centralnej CPU. Dla implementacji na GPU zostanie użyta technologia OpenCL. 

\paragraph{Struktura pracy} Na początku zostanie przedstawiony schemat algorytmu na GPU wraz z wyszczególnionymi kernelami. Następnie, przejdę do sposobu w jaki budowałem algorytm i jakie stosowałem optymalizacje w poszczególnych krokach. Po tym nastąpi porównanie otrzymanych wyników na GPU i CPU dla różnych grafów wejściwoych. Na samym końcu zostanie przedstawiona tabela z dokładnymi wynikami. 






\chapter{Schemat algorytmu na GPU}

\paragraph{Kernele} Aby przenieść algorytm z CPU na GPU należy stworzyć kilka kerneli obliczeniowych. Są one uruchamiane sekwencyjnie dla każdego wierzchołka (wirtualnego) grafu. Zostały zaimplementowane następujące kernele:
\begin{enumerate}
\item InitArray - inicjalizuje tablice
\item Forward - wykonuje fazę forward algorytmu
\item Middle - wypełnienie tablicy delta
\item Backward - wykonanie fazy backward algorytmu
\item Final - uaktualnienie tablicy wynikowej
\end{enumerate} 

Schemat algorytmu wygląda następująco:
 \begin{algorithm}[H]                                                            
   \caption{Schemat algorytmu z wyszczególnieniem użytych kerneli}\label{euclid}
   \begin{algorithmic}[1]
     \ForAll{$u \in V$}             
     \Comment{Dla każdego wierzchołka w grafie} 
     \State $\text{run Initarray(u)}$
     \State $\text{run Forward}$
     \State $\text{run Middle}$
     \State $\text{run Backward}$
     \State $\text{run Result(v)}$
     \EndFor                                            
  \end{algorithmic}                                                             
\end{algorithm} 

\paragraph{Maksymalna liczba sąsiadów wierzchołka}
Ponadto w algorytmie doświadczalnie wyznaczono maksymalną liczbę sąsiadów dla implementacji z wirtualnymi wierzchołkami. Liczba ta wynosi 32.


\chapter{Optymalizacje szybkości działania algorytmu oraz GPU}

\paragraph{Kroki podjęte podczas implementacji algorytmu} Poniższe zestawienie zawiera najważniejsze kroki podjęte podczas implementacji i optymalizacji algorytmu:

\begin{enumerate}
\item Implementacja bazowej wersji algorytmu na CPU
\item Przeniesienie algorytmu z CPU na GPU
\item Redukcja liczby kerneli
\item Próba skompresowania grafu 
\item Implementacja wersji rozszerzonej algorytmu (o wirtualne wierzchołki) na CPU
\item Dodanie do bazowego algortymu na GPU wirtualnych wierzchołków (zauważenie, że wierzchołki o dużej liczbie sąsiadów spowalniają działanie algorytmu, stąd znaczne przyspieszenie działania)
\item Dostrajanie algorytmu przy pomocy zmiany maksymalnej liczby sąsiadów dla danego wierzchołka
\item Końcowe optymalizcjie na GPU, np. cachowanie wartości globalnych w pętlach występujacych w kernelach (aby zmniejszyć liczbę odwołań do pamięci)
\end{enumerate}


\chapter{Porównanie wyników implementacji algorytmu na GPU i CPU}
\paragraph{GPU vs. CPU} W niniejszym rozdziale zostaną przedstawione wyniki z wykonania testów dla obydwu implementacji algorytmu. Testy te były wykonywane na karcie graficznej wspierającej technologię OpenCL \textbf{nVidia gtx 660}. Jest to karta graficzna gwarantująca przyspieszenie w stosunku do karty graficznej dostępnej na serwerach \textbf{nvidia1} oraz \textbf{nvidia2} rzędu \textbf{1,5-2x} (czas wykonania algorymtu Brandesa dla tej karty na danych \textbf{Facebook Small} to 3.669s, natomiast na kartach serwerów nvidia to 7.132s). Z kolei testy na układzie CPU zostały wykonane na procesorze \textbf{Intel Core i5-3230M}. Jest to procesor o częstotliwości taktowania zegara rzędu \textbf{2.60GHz-3.20GHz}.

\paragraph{Wyniki algorytmu na danych średniej wielkości} Na rysunku nr 4.1 został przedstawiony wykres porównujący czasy wykonania algorytmów dla obu platform obliczeniowych. Wynika z nich, że algorytm działa od 4 do 8 razy szybciej na karcie graficznej \textbf{nVidia gtx 660} (dla kart kraficznych na serwerze wydziałowym \textbf{nvidia} oznacza to przyspieszenie 2-4 krotne).



\begin{figure}[H]
  \centering
 % \framebox{\vbox to 4cm{\vfil\hbox to
 %     7cm{\hfil\tiny.\hfil}\vfil}}
  \includegraphics[width=15cm]{midium.png}
  \caption{Porównanie szybkości algorytmu Brandesa na GPU i CPU dla średnich grafów}
\end{figure}



\paragraph{Wyniki algorytmu na danych dużej wielkości} Z kolei na obrazku nr 4.2 przedstawiono wykres porównujący czasy wykonania algorytmów dla obu platform obliczeniowych, ale tym razem na dużych grafach wejściowych. Z przeprowadzonych testów wynika, że dla takich dancyh uzyskano przyspieszenie około 10-krotne. Obrazek został przedstawiony poniżej: 
\begin{figure}[H]
  \centering
 % \framebox{\vbox to 4cm{\vfil\hbox to
 %     7cm{\hfil\tiny.\hfil}\vfil}}
  \includegraphics[width=\textwidth]{large.png}
  \caption{Porównanie szybkości algorytmu Brandesa na GPU i CPU dla dużych grafów}
\end{figure}



\chapter{Tabela z wynikami}

\paragraph{Tabela z czasami} Poniższa tabela przedstawia czasy (w sekundach), jakie osiągnięto podczas przeprowadzania testów na algorytmie Brandesa:

\begin{table}[H]

 \begin{tabularx}{\textwidth}{ l | r }                                                      
  \hline                                                                            \input{dane}                                                         \hline                                                                          
  \end{tabularx}   
 \caption{Porównanie wyników algorytmu Brandesa na poszczególnych grafach}
\end{table}

\paragraph{Pliki z danymi wejściowymi}	Wszystkie powyższe testy będą dostępne do 30 czerwca 2014 roku na stronie internetowej  http://students.mimuw.edu.pl/\textasciitilde kl291649/pwir 

\chapter{Zawartość rozwiązania}
\paragraph{Pliki z kodami źródłowymi} Dołączony do opracowania folder z kodami źródłowymi zawiera:
\begin{enumerate}
\item Wersję algorytmu na GPU z zaimplementowanymi wirtualnymi wierzchołkami (pliki brandes*)
\item Wersję algorytmu na CPU z zaimplementowanymi wirtualnymi wierzchołkami (pliki cpubrandes*)
\item Raport z wyniki w formaci PDF
\item Plik Makefile
\end{enumerate}



\end{document}
